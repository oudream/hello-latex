\documentclass[12pt,article]{memoir}
\usepackage{amsmath}
\usepackage{xeCJK}
\setCJKmainfont[BoldFont={STHeiti},ItalicFont=STKaiti]{STSong}
\setCJKsansfont{STHeiti}
\setCJKmonofont{STFangsong}
\setCJKfamilyfont{zhsong}{STSong}
\setCJKfamilyfont{zhhei}{STHeiti}
\setCJKfamilyfont{zhkai}{STKaiti}
\setCJKfamilyfont{zhfs}{STFangsong}
\setCJKfamilyfont{zhli}{LiSu}
\setCJKfamilyfont{zhyou}{YouYuan}
\newcommand*{\songti}{\CJKfamily{zhsong}} % 宋体
\newcommand*{\heiti}{\CJKfamily{zhhei}}   % 黑体
\newcommand*{\kaishu}{\CJKfamily{zhkai}}  % 楷书
\newcommand*{\fangsong}{\CJKfamily{zhfs}} % 仿宋
\newcommand*{\lishu}{\CJKfamily{zhli}}    % 隶书
\newcommand*{\youyuan}{\CJKfamily{zhyou}} % 幼圆

\begin{document}

\chapter*{Matrices}

There are a couple of equivalent xxx-yyy ways to show matrices. Choose whichever you prefer, but please be consistent.

The \textsf{amsmath} package defines 中 the \texttt{matrix} environment and its friends:
\begin{align*}
	&
	\begin{matrix}
		a & b\\
		c & d
	\end{matrix}
	&
	&
	\begin{pmatrix}
		a & b\\
		c & d
	\end{pmatrix}
	&
	&
	\begin{Bmatrix}
		a & b\\
		c & d
	\end{Bmatrix}
	&
	&
	\begin{bmatrix}
		a & b\\
		c & d
	\end{bmatrix}
	&
	&
	\begin{vmatrix}
		a & b\\
		c & d
	\end{vmatrix}
	&
	&
	\begin{Vmatrix}
		a & b\\
		c & d
	\end{Vmatrix}
\end{align*}
%
The \textsf{memoir} class defines the \texttt{array}\footnote{Also see the \textsf{array} package if you're not using \textsf{memoir}} environment:
\begin{align*}
	&
	\begin{array}{cc}
		a & b \\
		c & d \\
	\end{array}
	&
	&
	\begin{array}({cc})
		a & b \\
		c & d \\
	\end{array}
	&
	&
	\begin{array}|{cc}|
		a & b \\
		c & d \\
	\end{array}
	&
	&
	\begin{array}\{{cc}\}
		a & b \\
		c & d \\
	\end{array}
\end{align*}
%
In both cases, the various enclosed matrices or arrays are simply shorthand ways of wrapping the naked environment with \verb|\left(...\right)| (or whatever delimiter) pairs. The \texttt{array} environment is more flexible because it supports the same column formatting arguments as the \texttt{tabular} environment.
\end{document}