
\documentclass[a4paper,12pt]{letter}

\name{Dr. L. User} % To be used for the return address on the envelope
\signature{Larry User} % Goes after the closing
\address{3245 Foo St.\\Gnu York\\USA}
% Alternatively, these may be set on an individual basis within the letter environment.

\makelabels % this command prints envelope labels on the final page of the document

\begin{document}


\begin{letter}{Sam Jones \\ Institute for Study\\ Princeton, N.J.} 

\opening{Deer John,} 

This is the basic \LaTeX\ letter class. It uses the classic formal letter layout, and is quite simple. It has most of the features of the \textsf{article} class, such as the enumerate, etc., environments and table/array capabilities.

This template is for the people who say ``oh, \TeX\ is clearly good for long reports and stuff, but writing a letter's heaps easier in Word''. I hope you'll agree the time saved \emph{not} setting up margins and indents, \textsl{ad nauseum} every time you need to write a letter, more than tips to balance in \LaTeX's favour.

\closing{Sincerely,}

\cc{Tinker\\Evers\\Chance} 

\encl{R\'esum\'e\\References} 

\ps{PS How 'bout this letter class?}
\ps{PPS Don't forget to look at \textsf{scrlttr2}, the KOMA-Script letter class, as well.}

\end{letter}




\begin{letter}{Albert Bensimon\\Adelaide, SA\\AUSTRALIA 5001}

\opening{To whom-ever it may or may not concern,}

You could keep all of your correspondence to a single person in a single document. If you really wanted. Don't forget to check out the envelope labels on the last page...but be warned: they're not very fancy!

If you don't want the next bit indented so much (or at all!), you can set the length with a \verb|\setlength{\longindentation}{0pt}|.

\setlength{\longindentation}{0pt}

\closing{Sincerely,}

\ps{\textsc{ps}: Don't use the other KOMA-Script letter class (\textsf{scrlettr})---it is obsolete!}

\end{letter}


\end{document}